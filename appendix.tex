\chapter{Code}\label{chapt:code}
\textbf{The code can be found at \url{https://nautatva.github.io/btp/code/}.}

\section{Summary}
\textbf{Language programmed}: Python 3.8 \bigskip \\
\textbf{General Idea}: The idea is to take an image and apply the concepts of super-resolution. This image, now with the enhanced resolution, is fed into a neural network to identify roads. The benefit of higher resolution means even smaller roads can now be identified. \bigskip \\
\textbf{Constraints}: Each side of the input image must have resolution divisible by 16 for super-resolution and 32 for road-detection.


\section{Notes}
\subsection{Installing GDAL}
A satellite image bit depth is usually 16 bits, and conventional images are 8-bit images. Thus, most of the common python libraries, such as PIL, openCV cannot be used due to information loss. We, therefore, use GDAL and scikit-image libraries to read and write images.

The easiest way to install GDAL is by using anaconda. Install anaconda from \url{https://www.anaconda.com/products/individual#Downloads} and run \mintinline{bash}{conda install -y gdal} on command line~(CMD) or terminal.


\subsection{Read and write an image}
Using GDAL can be tricky. Therefore, \texttt{`helper.py'} has the functions \texttt{read\_image} and \texttt{write\_image}. These take care of the possible corner-cases which may arise so that the libraries can be used easily without any problems. An extra function \texttt{`convertTo8Bit'} can be used to convert any image into 8 Bits.

\begin{minted}{python}
import helper

image_array = read_image('path_to_image')  # Reads image and stores the matrix in image variable; Lets say 16Bit image
write_image(image_array, 'output.png')  # Writes the image array to a file named output.png
convertTo8Bit('path_to_image', 'output_8bit.png')
\end{minted}


\subsection{Splitting large image into smaller chunks}
To ensure the image can be processed on a personal computer, we use the \texttt{`image.split()'} to divide the given image into multiple parts of a specified size.

\begin{minted}{python}
df = image.split('source', path(root_path, 'sliced'))  # df contains all the details of the smaller images. These small images are stored in the `sliced` directory
\end{minted}

\subsection*{Input}
\begin{itemize}
  \setlength\itemsep{1mm}
  \item The split function takes in an image or a folder path containing images and folder path for output as input.
  \item Optional parameters are SliceX and SliceY, whose default values are both 544. It is the dimension of the image we want for each chunk.
  \item Other optional parameters are StrideX and StrideY, which set the value of the image to be slid after cropping up. Default values are set to be 540 for both.
\end{itemize}

\subsection*{Output}
\begin{itemize}
  \setlength\itemsep{1mm}
  \item The function returns an array with the details of how the image is split.
  \item The files are always of dimension $SliceX\times SliceY$.
  \item The name convention for images saved is \newline
    nameOfImage\_\_InitialY\_InitialX\_sliceHeight\_sliceWidth\_padding\_TotalWidth\_TotalHeight \newline
    (with the required extension).
\end{itemize}


\subsection*{Example}
\begin{itemize}[label={}]
  \item\textbf{Input}:
  \begin{itemize}[{\textbullet}]
    \setlength\itemsep{1mm}
    \item \textbf{Name of Image}: coverleaf.jpg
    \item \textbf{Stride X}: 540
    \item \textbf{Stride Y}: 540
    \item \textbf{SliceX}: 544
    \item \textbf{SliceY}: 544
  \end{itemize}

  \item\textbf{Property of image}:
  \begin{itemize}[{\textbullet}]
    \setlength\itemsep{1mm}
    \item Width of image: 864
    \item Height of image: 864
  \end{itemize}

  \item\textbf{The output files are named}:
  \begin{itemize}[>]
    \setlength\itemsep{1mm}
    \item coverleaf\_\_0\_0\_544\_544\_0\_864\_864
    \item coverleaf\_\_0\_320\_544\_544\_0\_864\_864
    \item coverleaf\_\_320\_0\_544\_544\_0\_864\_864
    \item coverleaf\_\_320\_320\_544\_544\_0\_864\_864
  \end{itemize}
\end{itemize}
\section{Super Resolution}
For applying super-resolution on one image, use \texttt{`apply\_sr'}. When applying SR on multiple images, copy them into a folder, and use \texttt{`batch\_SR'}. You can download my trained weights from \href{https://nautatva.github.io/btp/weight/sr/}{here}.

\begin{minted}{python}
# Use depending on the need
from super_resolution import apply_SR, batch_SR


# SR on single image:
apply_SR('path_image_in', 'path_image_out', 'path_to_SR_weight')

# SR on multiple images:
batch_SR('folder_path_in', 'folder_path_out', 'extensions_of_files to be processed', 'path_to_SR_weight')
\end{minted}


\section{Road-detection}
Apply road detection using the \texttt{`find\_roads'} function. It takes in a folder and detects the road in every file of that folder. You can download my trained weights from \href{https://nautatva.github.io/btp/weight/road/}{here}.

\begin{minted}{python}
# Find roads
find_roads(source=path(root_path, "sr"), output=path(root_path, "roads"), weights="path_to_weights")  # Takes in the source folder sr, predicts roads in the image and writes the image in folder named roads.
\end{minted}

\section{Merge the split images}
Merging the images to get back the roads on the entire map is done by the \texttt{`image.merge'} function.
Set the scale to the factor of resolution change during super-resolution.

\begin{minted}{python}
merge('input_folder', 'output_folder', 'extension_of_files_to_be_joined', "scale_SR")
\end{minted}